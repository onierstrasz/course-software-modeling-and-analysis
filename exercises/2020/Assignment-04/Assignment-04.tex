\documentclass [11pt, a4wide, twoside]{article}

\usepackage{times}
\usepackage{epsfig}
\usepackage{ifthen}
\usepackage{xspace}
\usepackage{fancyhdr}
\usepackage[colorlinks,urlcolor=blue]{hyperref}
\usepackage{verbatim}
\usepackage{amsmath}
\usepackage{graphicx}

\usepackage{listings}
\usepackage{color}

\definecolor{dkgreen}{rgb}{0,0.6,0}
\definecolor{gray}{rgb}{0.5,0.5,0.5}
\definecolor{mauve}{rgb}{0.58,0,0.82}

\lstset{frame=tb,
  language=Java,
  aboveskip=3mm,
  belowskip=3mm,
  showstringspaces=false,
  columns=flexible,
  basicstyle={\small\ttfamily},
  numbers=none,
  numberstyle=\tiny\color{gray},
  keywordstyle=\color{blue},
  commentstyle=\color{dkgreen},
  stringstyle=\color{mauve},
  breaklines=true,
  breakatwhitespace=true
  tabsize=3
}

% solution switch
\newboolean{showsolution}
\setboolean{showsolution}{false} 

\usepackage{times}
\usepackage{epsfig}
\usepackage{ifthen}
\usepackage{xspace}
\usepackage{fancyhdr}
\usepackage{amsthm}
\usepackage{hyperref}



%layout
\topmargin      -5.0mm
\oddsidemargin  6.0mm
\evensidemargin -6.0mm
\textheight     215.5mm
\textwidth      160.0mm
\parindent      1.0em
\headsep        10.3mm
\headheight     12pt
\lineskip       1pt
\normallineskip 1pt

\newtheorem{mydef}{Definition}


%header
\lhead{Software Modeling and Analysis \\ 2016}

\rhead{Prof. Oscar Nierstrasz, Mohammad Ghafari, Nevena Milojkovi\'{c} and Yuriy Tymchuk}

\lfoot{page \thepage}
\rfoot{\today}
\cfoot{}

\renewcommand{\headrulewidth}{0.1pt}
\renewcommand{\footrulewidth}{0.1pt}

%enumeration
\newenvironment{myitemize}{%
     \begin{itemize}
     \setlength{\itemsep}{0cm}}
     {\end{itemize}}

\newenvironment{myenumerate}{%
     \begin{enumerate} \setlength{\itemsep}{0cm}}
     {\end{enumerate}}


%solution
\ifthenelse{\boolean{showsolution}}
   {  \newcommand{\solution}[1]{
   	\noindent\underline{\textbf{Answer:}}\\[2mm]
   	 \textsl{#1}
	 \vspace{10pt}
	 \normalsize
	}
  }
  {  \newcommand{\solution}[1]{} }


\pagestyle{fancy}

% ============================================================
% Markup macros for proof-reading
\usepackage{ifthen}
\usepackage[normalem]{ulem} % for \sout
\usepackage{xcolor}
\newcommand{\ra}{$\rightarrow$}
\newboolean{showedits}
\setboolean{showedits}{true} % toggle to show or hide edits
\ifthenelse{\boolean{showedits}}
{
	\newcommand{\ugh}[1]{\textcolor{red}{\uwave{#1}}} % please rephrase
	\newcommand{\ins}[1]{\textcolor{blue}{\uline{#1}}} % please insert
	\newcommand{\del}[1]{\textcolor{red}{\sout{#1}}} % please delete
	\newcommand{\chg}[2]{\textcolor{red}{\sout{#1}}{\ra}\textcolor{blue}{\uline{#2}}} % please change
}{
	\newcommand{\ugh}[1]{#1} % please rephrase
	\newcommand{\ins}[1]{#1} % please insert
	\newcommand{\del}[1]{} % please delete
	\newcommand{\chg}[2]{#2}
}
% ============================================================
% Put edit comments in a really ugly standout display
%\usepackage{ifthen}
\usepackage{amssymb}
\newboolean{showcomments}
\setboolean{showcomments}{true}
%\setboolean{showcomments}{false}
\newcommand{\id}[1]{$-$Id: scgPaper.tex 32478 2010-04-29 09:11:32Z oscar $-$}
\newcommand{\yellowbox}[1]{\fcolorbox{gray}{yellow}{\bfseries\sffamily\scriptsize#1}}
\newcommand{\triangles}[1]{{\sf\small$\blacktriangleright$\textit{#1}$\blacktriangleleft$}}
\ifthenelse{\boolean{showcomments}}
%{\newcommand{\nb}[2]{{\yellowbox{#1}\triangles{#2}}}
{\newcommand{\nbc}[3]{
 {\colorbox{#3}{\bfseries\sffamily\scriptsize\textcolor{white}{#1}}}
 {\textcolor{#3}{\sf\small$\blacktriangleright$\textit{#2}$\blacktriangleleft$}}}
 \newcommand{\version}{\emph{\scriptsize\id}}}
{\newcommand{\nbc}[3]{}
 \renewcommand{\ugh}[1]{#1} % please rephrase
 \renewcommand{\ins}[1]{#1} % please insert
 \renewcommand{\del}[1]{} % please delete
 \renewcommand{\chg}[2]{#2} % please change
 \newcommand{\version}{}}
\newcommand{\nb}[2]{\nbc{#1}{#2}{orange}}
\newcommand{\here}{\yellowbox{$\Rightarrow$ CONTINUE HERE $\Leftarrow$}}
\newcommand\rev[2]{\nb{TODO (rev #1)}{#2}} % reviewer comments
\newcommand\fix[1]{\nb{FIX}{#1}}
\newcommand\todo[1]{\nb{TO DO}{#1}}
\newcommand\on[1]{\nbc{ON}{#1}{red}} % add more author macros here
%\newcommand\XXX[1]{\nbc{XXX}{#1}{blue}}
%\newcommand\XXX[1]{\nbc{XXX}{#1}{brown}}
%\newcommand\XXX[1]{\nbc{XXX}{#1}{cyan}}
%\newcommand\XXX[1]{\nbc{XXX}{#1}{darkgray}}
%\newcommand\XXX[1]{\nbc{XXX}{#1}{gray}}
%\newcommand\XXX[1]{\nbc{XXX}{#1}{magenta}}
%\newcommand\XXX[1]{\nbc{XXX}{#1}{olive}}
%\newcommand\XXX[1]{\nbc{XXX}{#1}{orange}}
%\newcommand\XXX[1]{\nbc{XXX}{#1}{purple}}
%\newcommand\XXX[1]{\nbc{XXX}{#1}{red}}
%\newcommand\XXX[1]{\nbc{XXX}{#1}{teal}}
%\newcommand\XXX[1]{\nbc{XXX}{#1}{violet}}
%=============================================================
% ST80 listings macros
% Adapted from Squeak by Example book
%=============================================================
% If you want >>> appearing as right guillemet, you need these two lines:
%\usepackage[T1]{fontenc}
%\newcommand{\sep}{\mbox{>>}}
% Otherwise use this:
\newcommand{\sep}{\mbox{$\gg$}}
%=============================================================
%:\needlines{N} before code block to force page feed
\usepackage{needspace}
\newcommand{\needlines}[1]{\Needspace{#1\baselineskip}}
%=============================================================
%:Listings package configuration for ST80
\usepackage[english]{babel}
\usepackage{amssymb,textcomp}
\usepackage{listings}
% \usepackage[usenames,dvipsnames]{color}
% \usepackage[usenames]{color}
% \definecolor{source}{gray}{0.95}
\lstdefinelanguage{Smalltalk}{
%  morekeywords={self,super,true,false,nil,thisContext}, % This is overkill
  morestring=[d]',
  morecomment=[s]{"}{"},
  alsoletter={\#:},
  escapechar={!},
  literate=
    {BANG}{!}1
    {UNDERSCORE}{\_}1
    {\\st}{Smalltalk}9 % convenience -- in case \st occurs in code
    % {'}{{\textquotesingle}}1 % replaced by upquote=true in \lstset
    {_}{{$\leftarrow$}}1
    {>>>}{{\sep}}1
    {^}{{$\uparrow$}}1
    {~}{{$\sim$}}1
    {-}{{\sf -\hspace{-0.13em}-}}1  % the goal is to make - the same width as +
    {+}{\raisebox{0.08ex}{+}}1		% and to raise + off the baseline to match -
    {-->}{{\quad$\longrightarrow$\quad}}3
	, % Don't forget the comma at the end!
  tabsize=4
}[keywords,comments,strings]

\definecolor{source}{gray}{0.95}

\lstset{language=Smalltalk,
	basicstyle=\sffamily,
	keywordstyle=\color{black}\bfseries,
	% stringstyle=\ttfamily, % Ugly! do we really want this? -- on
	mathescape=true,
	showstringspaces=false,
	keepspaces=true,
	breaklines=true,
	breakautoindent=true,
	backgroundcolor=\color{source},
	lineskip={-1pt}, % Ugly hack
	upquote=true, % straight quote; requires textcomp package
	columns=fullflexible} % no fixed width fonts
% In-line code (literal)
% Normally use this for all in-line code:
\newcommand{\ct}{\lstinline[mathescape=false,backgroundcolor=\color{white},basicstyle={\sffamily\upshape}]}
% In-line code (latex enabled)
% Use this only in special situations where \ct does not work
% (within section headings ...):
\newcommand{\lct}[1]{{\textsf{\textup{#1}}}}
% Code environments
\lstnewenvironment{code}{%
	\lstset{%
		% frame=lines,
		frame=single,
		framerule=0pt,
		mathescape=false
	}
}{}

% Useful to add a matching $ after code containing a $
% \def\ignoredollar#1{}
%=============================================================
\newcommand{\ie}{\emph{i.e.}\xspace}
\newcommand{\eg}{\emph{e.g.}\xspace}
\newcommand{\etal}{\emph{et al.}\xspace}
% ============================================================

\definecolor{mircolor}{rgb}{0.4,0.6,0.2}
\newcommand\ml[1]{\nbc{ML}{#1}{mircolor}}
\usepackage{wasysym}
\newcommand\yesml[1]{\nbc{ML {\textcolor{yellow}\sun}}{#1}{mircolor}}

\begin{document}
\section*{\ifthenelse{\boolean{showsolution}}{\textcolor{red}{Solution}\\}{}Assignment 04 --- 07.10.2020-- v1.0b\\Smalltalk: Reflection}

\textcolor{red}{Please submit this assignment by email to \href{mailto:pascal.gadient@inf.unibe.ch}{pascal.gadient@inf.unibe.ch} \underline{before} 14. October 2020, 10:15am.}

\subsection*{Exercise 1 - Hierarchy traversal (1 pt)}
Write a method that finds the class with the longest inheritance chain among all Smalltalk classes in the GT programming environment.\\\\
\noindent\textit{NB: To access all classes of Smalltalk, you can use \texttt{SystemNavigation default allClasses}.}
\solution{\texttt{(((SystemNavigation default allClasses collect:\newline
\hspace*{0.5cm}[:eachClass | eachClass -> eachClass classDepth]) sorted:\newline
\hspace*{1.0cm}[ :a :b | a value > b value ]) asOrderedDictionary) keys first}\\\newline
\texttt{keys first} will point to the class with the longest inheritance chain, \ie the class \texttt{PRYoutubeSemLinkTest} with a class depth of 13 elements. Please note that there exists one more class at the same hierarchy level: \texttt{PRWikipediaSemLinkTest}.}

\subsection*{Exercise 2 - Method overrides (2 pts)}
Write a method to find all methods that override an abstract method in GT.
\solution{\texttt{(SystemNavigation default allMethods select: \#isAbstract) flatCollect:\newline
\hspace*{0.5cm}[ :m | ((m methodClass allSubclasses flatCollect: \#methods) select:\newline
\hspace*{1.0cm}[ :n | m selector = n selector ]) reject: \#isAbstract ]}}

\subsection*{Exercise 3 - Query methods (2 pts)}
Write a method that finds all classes with at least one query method in GT.\\\\
\textit{NB: Query methods test a property of an object. Such methods are prefixed with \texttt{is}, \texttt{was} or \texttt{will}.}
\solution{\texttt{SystemNavigation default allClasses select:\newline
\hspace*{0.5cm}[ :class | class methodDict keys anySatisfy:\newline
\hspace*{1.0cm}[ :sel | (\textquotesingle is*\textquotesingle~match: sel) | \newline
\hspace*{1.0cm}(\textquotesingle was*\textquotesingle~match: sel) | (\textquotesingle will*\textquotesingle~match: sel)]\newline
\hspace*{0.5cm}].}}

\ifthenelse{\boolean{showsolution}}{\pagebreak}{}

\subsection*{Exercise 4 - Root methods (2 pts + 2 pts BONUS)}
\begin{enumerate}[i)]
\item Find all root methods in GT.\\\\
\textit{NB: A ``root method'' is a method whose selector has been implemented in a class, such that the superclasses of that class do not understand it.}
\solution{\texttt{introducedMethods := [ :class | class superclass\newline
\hspace*{0.5cm}ifNil: [ class methods ]\newline
\hspace*{0.5cm}ifNotNil: [ class methods select:\newline
\hspace*{1.0cm}[ :met | (class canUnderstand: met selector) \& \newline
\hspace*{1.0cm}(class superclass canUnderstand: met selector) not ]]].\newline\newline
SystemNavigation default allClasses flatCollect:\newline
\hspace*{0.5cm}[:cl | introducedMethods value: cl].}}
\item (BONUS) Find all duck-typed methods in GT.\\\\ 
\textit{NB: Duck-typed methods have the same selector but are not related by inheritance.
That is, after finding all root methods, find those with the same selector.}
\solution{\texttt{rootMethods sort: [ :m1 : m2 | m1 selector \textless m2 selector ].\\
rootSelectors := (rootMethods collect: \#selector).\\
duckSelectors := (rootSelectors asBag\\
\hspace*{0.5cm}removeAll: rootSelectors asSet; yourself) asSet.\\
rootMethods select: [ :met | duckSelectors includes: met selector]}}
\end{enumerate}

\vspace{1.5cm}
\begin{center}\center{\large{\emph{\textbf{Please continue reading on the next page.}}}\newpage}\end{center}

\subsection*{Exercise 5 - Dynamic coding (3 pts)}
\emph{This exercise carries on with exercise 3 of the second assignment.
As stated before, you have to download the \texttt{CallGraph} code from Github, and you must store the \texttt{Calls.txt} file in the same folder as the GT image file.}\\\\
Your task is to redefine the method \texttt{doesNotUnderstand: aMessage} in the provided class \texttt{Call}.
The redefined method should \underline{dynamically} create an instance variable and a method that returns the number of arguments.
In order to achieve that, you are supposed to follow these three steps:\\\\
\textbf{Step 1:} Within the method, add \underline{dynamically} the instance variable \texttt{numberOfArguments} to the class \texttt{Call} if it does not already exist.\\\\
\textbf{Step 2:} Within the method, add \underline{dynamically} the method below to the class \texttt{Call}. Since you are adding that method during run time, you must compile it from a String representation.\\\\
\texttt{numberOfArguments\\
numberOfArguments := args size.\\
\caret~numberOfArguments.}\\\\
\textbf{Step 3:} So far, the initial execution does nothing but enable the \texttt{numberOfArguments} method. Hence, we have to resend the initial message to \texttt{self}.\\\\
You can test your implementation by executing the following code:\\\\
\texttt{(CallGraph fromFile: \textquotesingle Calls.txt\textquotesingle) calls \\
\hspace*{0.5cm}collect: [ :each | each numberOfArguments]}\\\\
After you successfully implemented the \texttt{doesNotUnderstand} method, the statement will print the number of arguments for every call in the call graph (without raising a \texttt{doesNotUnderstand} error).

\vspace{1.5cm}
\ifthenelse{\boolean{showsolution}}{\begin{center}\center{\large{\emph{\textbf{Please continue reading on the next page.}}}\newpage}\end{center}}{}

\solution{\texttt{doesNotUnderstand: aMessage\newline
|messageName|\newline
messageName := aMessage selector asString.\newline\newline
messageName = \textquotesingle numberOfArguments\textquotesingle\newline
ifTrue: [ \newline
\hspace*{0.5cm}(self class allInstVarNames includes: \textquotesingle numberOfArguments\textquotesingle)\newline
\hspace*{0.5cm}ifFalse: [ self class addInstVarNamed: \textquotesingle numberOfArguments\textquotesingle].\newline\newline
\hspace*{0.5cm}self class compile:\newline
\hspace*{0.5cm}messageName, String cr, \newline
\hspace*{0.5cm}messageName, \textquotesingle~:= args size.\textquotesingle, String cr, \newline
\hspace*{0.5cm}\textquotesingle\caret\textquotesingle, messageName, \textquotesingle .\textquotesingle .\newline\newline
\hspace*{0.5cm}\caret~aMessage sendTo: self.\newline
].}}
\end{document}