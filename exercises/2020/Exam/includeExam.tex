\usepackage{verbatim}
\usepackage{times}
\usepackage{ifthen}
\usepackage{fancyhdr}
\usepackage{graphicx}
\usepackage{setspace}
\usepackage{listings}
\usepackage{tabto}
\usepackage[colorlinks,urlcolor=black,linkcolor=black,urlcolor=blue]{hyperref}
\usepackage{lastpage}
\usepackage{subfig}
\usepackage{amsmath}
\usepackage{enumerate}
\usepackage{xspace}
\usepackage{textcomp}
\usepackage[euler-digits,euler-hat-accent]{eulervm}

\newcommand{\caret}{\hspace*{0.125cm}$\hat{}$\hspace*{0.05cm}}

\lstset{
  backgroundcolor=\color{white},   % choose the background color; you must add \usepackage{color} or \usepackage{xcolor}
  basicstyle=\footnotesize\ttfamily,        % the size of the fonts that are used for the code \tiny \scriptsize
  breakatwhitespace=false,         % sets if automatic breaks should only happen at whitespace
  breaklines=true,                 % sets automatic line breaking
  captionpos=b,                    % sets the caption-position to bottom
  mathescape=true,
  commentstyle=\color{black},    % comment style
  deletekeywords={...},            % if you want to delete keywords from the given language
  escapechar=\@
  escapeinside={\%*}{*)},          % if you want to add LaTeX within your code
  extendedchars=true,              % lets you use non-ASCII characters; for 8-bits encodings only, does not work with UTF-8
  keepspaces=true,                 % keeps spaces in text, useful for keeping indentation of code (possibly needs columns=flexible)
  keywordstyle=\color{black},       % keyword style
  language=Java,                 % the language of the code
  morekeywords={*,...},            % if you want to add more keywords to the set
  numbers=none,                    % where to put the line-numbers; possible values are (none, left, right)
  numberstyle=\tiny, % the style that is used for the line-numbers
  rulecolor=\color{black},         % if not set, the frame-color may be changed on line-breaks within not-black text (e.g. comments (green here))
  showspaces=false,                % show spaces everywhere adding particular underscores; it overrides 'showstringspaces'
  showstringspaces=false,          % underline spaces within strings only
  showtabs=false,                  % show tabs within strings adding particular underscores
  stepnumber=2,                    % the step between two line-numbers. If it's 1, each line will be numbered
  stringstyle=\color{black},     % string literal style
  tabsize=2,                       % sets default tabsize to 2 spaces
  xleftmargin=.25in,
  title=\lstname                   % show the filename of files included with \lstinputlisting; also try caption instead of title
}

% solution switch
\newboolean{showsolution}
\setboolean{showsolution}{true} 

%layout
\topmargin      -5.0mm
\oddsidemargin  6.0mm
\evensidemargin -6.0mm
\textheight     215.5mm
\textwidth      160.0mm
\parindent      1.0em
\headsep        10.3mm
\headheight     12pt
\lineskip       1pt
\normallineskip 1pt

%header
\lhead{SMA: Software Modeling and Analysis\\A2020}

\rhead{Prof. Dr. Oscar Nierstrasz\\Pascal Gadient, Pooja Rani}

\lfoot{page \hspace{0.03cm} \thepage \hspace{0.11cm} of \hspace{0.09cm} \pageref*{LastPage}}
\rfoot{16 December 2020}
%\rfoot{\today}
\cfoot{}

\renewcommand{\headrulewidth}{0.1pt}
\renewcommand{\footrulewidth}{0.1pt}

\def\xexercise{\fontsize{12}{10}\fontseries{bx}\selectfont}
\def\xnormal{\fontseries{m}\fontshape{n}\selectfont}

\newcounter{exnum}
\newcommand{\exercise}[1]{%
	{\addtocounter{exnum}{1}\vskip 0.8cm{\xexercise \noindent Exercise \arabic{exnum} -- #1} \xnormal} \vskip 0.3cm}
\newcommand{\aufgabe}[1]{
	{\addtocounter{exnum}{1}\vskip 0.8cm{\xexercise \noindent Aufgabe \arabic{exnum} - #1} \xnormal} \vskip 0.3cm}

%enumeration
\newenvironment{myitemize}{%
     \begin{itemize}
     \setlength{\itemsep}{0cm}}
     {\end{itemize}}

\newenvironment{myenumerate}{%
	\renewcommand{\labelenumi}{\alph{enumi})}
	\renewcommand{\labelenumii}{\arabic{enumii})}
     \begin{enumerate} \setlength{\itemsep}{0cm}}
     {\end{enumerate}}


%solution
\ifthenelse{\boolean{showsolution}}
   {  \newcommand{\solution}[1]{
   	\noindent\underline{\textcolor{red}{\textbf{Answer:}}}\\[2mm]
   	 \textsl{\textcolor{red}{#1}}
	 \vspace{10pt}
	 \normalsize
	}
  }
  {  \newcommand{\solution}[1]{} }


\pagestyle{fancy}


\newcommand{\ie}{\emph{i.e.},\xspace}
\newcommand{\eg}{\emph{e.g.},\xspace}
\newcommand{\etc}{\emph{etc.}\xspace}
